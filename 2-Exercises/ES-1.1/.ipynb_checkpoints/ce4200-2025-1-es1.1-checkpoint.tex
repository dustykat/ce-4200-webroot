\documentclass[12pt]{article}
\usepackage{geometry}                % See geometry.pdf to learn the layout options. There are lots.
\geometry{letterpaper}                   % ... or a4paper or a5paper or ... 
%\geometry{landscape}                % Activate for for rotated page geometry
\usepackage[parfill]{parskip}    % Activate to begin paragraphs with an empty line rather than an indent
\usepackage{daves,fancyhdr,natbib,graphicx,dcolumn,amsmath,lastpage,url}
\usepackage{amsmath,amssymb,epstopdf,longtable}
\usepackage{paralist} 
\usepackage[final]{pdfpages}
\DeclareGraphicsRule{.tif}{png}{.png}{`convert #1 `dirname #1`/`basename #1 .tif`.png}
\pagestyle{fancy}
\lhead{CE 4200 -- Professional Engineering Practice Issue}
\rhead{SPRING 2025}
\lfoot{EXERCISE 1.1}
\cfoot{}
\rfoot{Page \thepage\ of \pageref{LastPage}}
\renewcommand\headrulewidth{0pt}

\begin{document}

\begin{center}
{\textbf{{ CE 4200 -- Professional Engineering Practice Issues} }}
\end{center}

\section*{\small{Exercise 1.1}} 
\emph{The Physical Body as an Extension of the Mind: Analyzing the Role of the Psychomotor Domain in Engineering Knowledge}


\section*{\small{Purpose}} 
\begin{itemize}
\item Access to the class website and upload completed exercises as .PDF files.
\item Use web resources and supplied readings to self-teach about knowledge domains.
\item Develop familiarity with current concepts of the "mind" and "body" interactions.
\item Become familiar enough with BOK3 to critique percieved omissions
\end{itemize}

\section*{\small{Exercise}}
Explore in an essay, the concept of the physical body (arms, legs, and other motor functions) as an integral part of the ``mind,'' in addition to the mental processes traditionally associated with the brain. Your essay should consider how the physical and mental aspects of the human experience are interconnected and essential for holistic learning and professional performance. Then critically evaluate whether the omission of the psychomotor domain in the ASCE Body of Knowledge 3rd Edition (BOK3) represents an oversight or ignorance of its importance in civil engineering education and practice.

\section*{\small{Essay Guidelines}}
\begin{itemize}
    \item \textbf{Length}: 800-1000 words (roughly 4 pages, exclusive of references)
    \item \textbf{Structure}:
    \begin{enumerate}
        \item \textbf{Introduction}:
        \begin{itemize}
            \item Define the concept of the ``mind'' as including both mental and physical components.
            \item Briefly introduce the three knowledge domains (cognitive, affective, and psychomotor) and their relevance to professional practice.
            \item State your thesis: Why the psychomotor domain is (or isn’t) crucial to the outcomes of civil engineering education.
        \end{itemize}
        
        \item \textbf{Body}:
        \begin{enumerate}
            \item \textbf{Section 1}: Explain how the physical body contributes to the functioning of the mind.
            \begin{itemize}
                \item Use examples from engineering practice, such as hands-on problem-solving, tool usage, or fieldwork.
                \item Discuss how physical actions often enhance mental understanding (e.g., learning through doing).
            \end{itemize}
            \item \textbf{Section 2}: Evaluate the BOK3 outcomes and their focus on cognitive and affective domains.
            \begin{itemize}
                \item Identify areas where the psychomotor domain might naturally complement or enhance the existing outcomes.
                \item Analyze potential reasons for its omission (e.g., historical focus, disciplinary biases).
            \end{itemize}
            \item \textbf{Section 3}: Present arguments for or against including the psychomotor domain in future revisions of the BOK.
            \begin{itemize}
                \item Consider the impact of psychomotor skills on safety, effectiveness, and creativity in engineering.
            \end{itemize}
        \end{enumerate}
        
        \item \textbf{Conclusion}:
        \begin{itemize}
            \item Summarize your main points.
            \item Reflect on how integrating the psychomotor domain might benefit engineering education and practice.
            \item Pose a thought-provoking question about the future of engineering education.
        \end{itemize}
    \end{enumerate}
\end{itemize}

\section*{\small{Evaluation Criteria\footnote{The essay will be evaluated by a narrow domain AI agent, with instructions to identify the listed items.  The AI agent will also have access to BOK3 as well as most of the references in the notes.}}}
\begin{itemize}
    \item \textbf{Depth of Analysis}: Does the essay thoughtfully explore the integration of physical and mental faculties and its relevance to engineering?
    \item \textbf{Critical Thinking}: Does the essay provide a well-reasoned argument for or against the inclusion of the psychomotor domain in BOK3?
    \item \textbf{Clarity and Organization}: Is the essay well-structured, with a clear thesis and logical flow?
    \item \textbf{Use of Examples}: Are the arguments supported by relevant examples from engineering education and practice?
    \item \textbf{Engagement with the BOK3}: Does the essay demonstrate a clear understanding of the BOK3 and its outcomes?
\end{itemize}

\section*{\small{Submission Requirements\footnote{The essay will be evaluated using a narrow domain AI agent; therefore structure and file type are essential}}}
\begin{itemize}
    \item \textbf{Format}: Typed, double-spaced, 12-point Times New Roman font, 1-inch margins.
    \item \textbf{Submission}: Upload your essay as a PDF to the course’s online portal (Blackboard).
\end{itemize}

\section*{\small{Additional Notes}}
\begin{itemize}
    \item Citations should be in a consistent format; for on-line resources, include valid DOI or URL.
\end{itemize}

\end{document}

