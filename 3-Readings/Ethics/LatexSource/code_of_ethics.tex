\section*{NSPE Code of Ethics}
The NSPE code of ethics is actually pretty simple in structure.  There is the preamble that describes the purpose of the code which is to safeguard life, health, property,\footnote{Duty Ethics} promote the public good, \footnote{Utility Ethics} and maintain high integrity\footnote{Value Ethics - I am assuming that the NSPE is focused on the individual engineer's character - recall that to be licensed one question that our SER reviewers must answer is if they think the candidate is moral enough to be an engineer - a question of character.}.

The next part of the code is the fundamental canons.  These are simply a set of rules that engineers are expected to follow.  The rules deal with only three fundamental issues: Engineers are obligated to uphold public welfare, obligated to be competent, and are obligated to be honest.  The individual rules paraphrased are:
\begin{enumerate}
\item  Public welfare is paramount.\footnote{Like the "prime directive" in the Star Trek series.} The code provides guidelines for how to act in generalized situations.  It describes correct engineering behavior.  Notes that engineers have an obligation to report illegal and unethical behavior.
\item Obligated to work in areas of competence.  Both the NSPE code and the Texas Engineering Practices Act observe that experience and/or education confers �competence�.  Self-education would qualify (hard to prove).  Judgment is expected.  Project management is covered (modern management theory).  
\item Tell the truth; Requires disclosure if paid for an opinion (in advance of the opinion). 
\item Preserve the engineer-client privilege (confidence, secrecy, competitive advantage).  Disclose conflict of interests -- esp. applies to public employees.
\item Be honest; Respect other engineers.  Do not misrepresent ones own abilities or downplay a competitors abilities.  This rule does not preclude pointing out a competitors experience (which is a matter of business records), but one truly has no way of determining a competitors ability.  Don�t accept things that are or look like a bribe, kickback, payoff etc. to obtain a job.  You can still meet with your friends for lunch, discuss work, and even take turns paying the bill.  
\item Uphold your obligation to the profession.  One obligation is to teach others - thus that competitive advantage above eventually should be shared.  Oddly enough improving your competitors capabilities not only promotes the public good, it is probably good business in the long run.  
\end{enumerate}

\section*{Texas Engineering Practices Act}
The act is fairly long 70 or so pages.  The ethics and professional development component is a small but important part of the Act.  A majority of the Act establishes legislative authority, how it is funded, etc.  The important parts for this seminar are that the Act defines the ''practice of engineering'', defines ''licensure qualifications'' and defines ''misconduct.''  Some of the relevant sections for this seminar are listed below.

\S 1001.004 Establishes the purpose of the Act:  Promote the public good, improve quality of life, property, economy, and security of the state and the nation.

\S 1001.$...$ Specific definitions; defines a public work, sets dollar values on projects that need licensed engineers to perform services; establishes fees for various license related activities.

\S 1001.210 Establishes the continuing education program - this seminar is relevant in the context of this section.  The continuing education component is generous - there is no reason why a person cannot accumulate the required annual development hours the things that count include:

\begin{enumerate}
\item A course sponsored by an institution of higher education, a professional, or a trade organization
\item A seminar, tutorial, short course, correspondence course, videotaped course, or televised course.
\item Participating in an in-house course sponsored by a corporation or other business entity
\item Teaching a course described by the Act
\item Publishing an article, paper, or book on the practice of engineering
\item Making or attending a presentation at a meeting of a technical or engineering management society or organization or  writing a paper presented at such a meeting; 
\item Participating in the activities of a professional society or association, including serving on a committee of the organization
\item Engaging in self-directed study (up to 5 hours).
\end{enumerate}

At least 1 hour needs to be ethics related, reading the act each year would count and could be logged in the self-directed component.

\S 1001.301. A license is required to practice engineering and/or use following titles: 
\begin{enumerate}
\item engineer
\item professional engineer
\item licensed engineer
\item registered engineer
\item registered professional engineer
\item licensed professional engineer
\item engineered ... (used as verb/adverb?) in describing some work.  This particular word is in that grey area - my laptop is engineered, but I am sure the State of Texas had no say in who, where, or how.  On the other hand my sewage collection system was engineered and that was well within the provisions of the act.
\end{enumerate}

The next section delegates state authority to political subdivisions of the state.

\S 1001.402. Enforcement by Certain Public Officials.
A public official of the state or of a political subdivision of the state who is responsible for enforcing laws that affect the practice of engineering may accept a plan, specification, or other related document only if the plan, specification, or other document was prepared by an engineer, as evidenced by the engineer�s seal. 

The next section deals with engineering requirements for a public work where consequences involve health, welfare, and safety.

\S 1001.407. Construction of Certain Public Works 
The state or a political subdivision of the state may not construct a public work involving engineering in which the public health, welfare, or safety is involved, unless: 

(1) the engineering plans, specifications, and estimates have been prepared by an engineer; and 

(2) the engineering construction is to be performed under the direct supervision of an engineer. 

The next section contains detailed definitions of various items - of particular interest is negligence and incompetence.

\S131.81 Definitions
Defines meanings of acronyms (ABET) etc. and specific terms such as �graduate engineer� etc. 

Gross negligence - Any \textbf{willful} or \textbf{knowing conduct}, or pattern of conduct, which includes but is not limited to conduct that demonstrates a \textbf{disregard} or indifference to the \textbf{rights}, \textbf{health, safety, welfare, and property} of the public or clients. 

Gross negligence \textbf{may} result in financial loss, injury or damage to life or property, but such results \textbf{need not occur} for the establishment of such conduct. 

This last sentence means that "no harm no foul" is not a defense.

Incompetence - An \textbf{act} or \textbf{omission} of malpractice which may include but is not limited to \textbf{recklessness} or \textbf{excessive errors}, omissions or failures in the license holder�s record of professional practice; or an act or omission in connection with a disability which includes but is not limited to mental or physical disability or addiction to alcohol or drugs as to \textbf{endanger} health, safety and interest of \textbf{the public} by impairing skill and care in the provision of professional services. 

The quote "Don't drink and derive" has more significance than a funny pun on a t-shirt.

Summary: Negligence is disregard for rights, safety, etc.   Harm need not occur for negligence.  Incompetence is recklessness, inability, impaired or missing skill that endangers the public - notice that the first part includes poor record keeping (that endangers the public).   Also observe that the words explicitly acknowledge that engineers may make mistakes - it demands that there be few of these and that we behave in good faith.  Engineering involves risk of failure and advancement of the state of knowledge may indeed lead to failure - the Act (in my opinion) acknowledges such and simply demands that engineers do not willfully disregard safety, keep good enough records so we (the profession) can learn from mistakes, and that we do not make too many at once.

\S137.37 Sealing Misconduct.  A license holder shall be guilty of misconduct and subject to disciplinary action if the license holder: 

(1) knowingly signs or seals any engineering document or product if its use or implementation may endanger the health, safety, property or welfare of the public.

(2) signs or affixes a seal on any document or product when the license is inactive or has been revoked, suspended, or has expired. 

(3) alters a sealed document without proper notification to the responsible license holder.

The last part of this section of the seminar lists the some of the section titles of the ethics section of the Act.  If you compare these to the NSPE code of ethics the parallel is obvious (order is different!).

\S137.55 Engineers Shall Protect the Public

\S137.57 Engineers Shall be Objective and Truthful

\S137.59 Engineers� Actions Shall Be Competent

\S137.61 Engineers Shall Maintain Confidentiality of Clients 

\S137.63 Engineers� Responsibility to the Profession